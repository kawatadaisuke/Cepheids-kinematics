\documentclass[11pt,a4paper]{article}
\RequirePackage{graphicx}
%\usepackage{eurosym}
\usepackage[small,compact]{titlesec}



% hyperref is great, but remember to define or include the macros used
% in ADS's bibtex entry
\usepackage{hyperref}
\input{/Users/dkawata/work/paper/journals}
\usepackage{natbib}
\bibliographystyle{/Users/dkawata/work/paper/mnras}

\textwidth=17.75cm
\textheight=25.0 cm
\topmargin=-2.0 cm
\oddsidemargin=-0.75cm
\evensidemargin=-0.75cm

%
% Start of document
%
\begin{document}
\title 
% Title
{\bf \normalsize
 Axisymmetric model fitting to Cepheids data
 }
%
\author{ \small Daisuke Kawata (MSSL, UCL)}
\date{}

\maketitle

\small
\begin{center}
{\bfseries Abstract}
\end{center}
\begin{quotation}
\begin{small}
\vspace{-5pt}
% Abstract
  Summary of axisymmetric disk model adapted from \citet{baabbdc12}, to apply it to the Cepheids data. 
%
\end{small}
\end{quotation}

%
\bigskip
%

\section{Axisymmetric disc kinematic model}

Following \citet{baabbdc12}, we compute the mean and dispersion of $V_{\rm los}$ expected in an axisymmetric Galactic disc model in the Galactic rest frame, and compare with the observational data. We also compute the same values for velocity in Galactic longitude direction, $V_{\rm glon}$, and include them for likelihood function. The model assumes a Gaussian velocity dispersion in the rotation and radial direction, and zero mean radial velocity. 

In the Galactic rest frame, the mean rotation velocity at the position of the star, $\overline{V_{\phi}}$, can be projected to the line of sight velocity, $V_{\rm los}$ from the observer, i.e. the position of the Sun as $V_{\rm m,los}=\overline{V_{\phi}} \sin(\phi+l)$, where $\phi$ is the angle between the line from the Galactic centre toward Sun and the one toward the position of the star, positive in clockwise, and $l$ is Galactic longitude. Angle $\phi$ can be calculated with $\phi=sign(l-180) {\rm acos}((R_0^2+R^2-d_{xy}^2)/(2 R_0 R)$, where $R_0$ is the Galactocentric radius of the Sun, R is the Galactocentric radius of the star, and $d_{xy}$ is the distance from the Sun to the star in the plane. Similarly, the velocity dispersion in the rotation direction can be projected to the line of sight direction, like $\sigma_{\rm los, \sigma_{\phi}}=\sigma_{\phi} \sin(\phi+l)$. On the other hand, the radial velocity dispersion is projected to the line of sight direction by $\sigma_{\rm los, \sigma_{R}}=\sigma_R \sin(\phi+l)$. Then, the expected velocity dispersion for the line-of-sight direction is $\sigma_{\rm m,los}=\sqrt{\sigma_{\rm los, \sigma_{\phi}}^2+\sigma_{\rm los, \sigma_{R}}^2}=\sqrt{\sigma^2(1+\sin^2(\phi+l) (X^2-1))}$, where $X^2=\sigma_{\phi}^2/\sigma_{R}^2$. These are shown in Section 3.2 of \citet{baabbdc12}.

Following the same strategy, we can derive the mean and dispersion of $V_{\rm glon}$ as follows. $V_{\rm m, glon}=\overline{V_{\phi}}\cos(\phi+l)$ and $\sigma_{\rm glon,\sigma_R}=\sigma_R \cos(\phi+1)$. Hence, $\sigma_{\rm m, glon}^2=\sigma_R^2(1+\cos^2(\phi+l) (X^2-1))$.

The mean rotation velocity, $\overline{V_{\phi}} (R)$, is calculated from asymmetric drift, $V_{\rm a}$ as $\overline{V_{\phi}}(R)=V_{\rm c} (R)-V_{\rm a}(R)$, where $V_c(R)$ is the circular velocity at radius, $R$. Following \citet{baabbdc12}, asymmetric drift is calculated by 
\begin{equation}
V_{\rm a}(R)=\frac{\sigma_R^2(R)}{2 V_{\rm c}(R)} \left[X^2-1+R\left(\frac{1}{h_R}+\frac{2}{h_{\sigma}}\right)\right].
\end{equation}
Here, $h_R$ and $h_{\sigma}$ are radial scale length of surface mass density and the radial scale length of the radial velocity dispersion profile, respectively. As discussed in \citet{baabbdc12}, the Galactic parameters we are interested in are not sensitive to these parameters. Hence, we fix $h_R=3.0$ and $h_{\sigma}=200$, i.e. a flat velocity dispersion. We also take into account the slope of the circular velocity at $R_0$, $d V_{\rm c}(R_0)/dR$, when we calculate $V_c(R)$. 

In observational data, we have the line-of-sight velocity, $V_{\rm los}^{\rm helio}$, and Galactic longitudinal velocity, $V_{\rm glon}^{\rm helio}$, with respect to the Solar motion. Using the Solar radial and rotation velocities, $V_{\rm R,\odot}$ (outward motion is positive) and $V_{\rm \phi,\odot}$ (clock-wise rotation is positive, $V_{\rm \phi,\odot}=V_{\rm c}+V_{\odot}$), these velocities can be converted to the Galactic rest-frame velocities as follows.
\begin{eqnarray}
V_{\rm o,los} & = & V_{\rm los}^{\rm helio} - V_{R,\odot} \cos l + V_{\rm \phi,\odot} \sin l, \\
V_{\rm o,glon} & = & V_{\rm glon}^{\rm helio} + V_{R,\odot} \sin l + V_{\rm \phi,\odot} \cos l.
\end{eqnarray}

\section{MCMC parameter probabilities}

We consider the posterior probability to find the marginalised probability distribution function of our model parameters.
\begin{equation}
 p(\theta_m=V_{\rm c}(R_0), V_{\phi,\odot}, V_{R,\odot}, \sigma_{\rm R}(R_0), X^2, R_0, V_{\rm los,sys}, dV_{\rm c}(R_0)/dR| \mathcal{D})
 = \mathcal L(\mathcal{D}|\theta_m) \times Prior,
\end{equation} 
where $\mathcal{D}$ describes the observational data, and $\theta_m$ corresponds to the model parameters. We run MCMC for $\ln p$. We also model a possible systematic motion of $V_{\rm los,sys}$ for the radial line-of-sight velocity, and take this into account as
\begin{equation}
V_{\rm o,los} = V_{\rm los}^{\rm helio}-V_{\rm los,sys} - V_{R,\odot} \cos l + V_{\rm \phi,\odot} \sin l.
\end{equation}
Likelihood function is described with 
\begin{equation}
\mathcal{L}=\prod_i^N \frac{1}{2 \pi \sigma_{\rm m,los,i} \sigma_{\rm m,los,i}} 
 \exp\left(-\frac{(V_{\rm o,los,i}-V_{\rm m,los,i})^2}{2 \sigma_{\rm m,los,i}^2}\right)
  \exp\left(-\frac{(V_{\rm o,glon,i}-V_{\rm m,glon,i})^2}{2 \sigma_{\rm m,glon,i}^2}\right).
\end{equation}
We found that $R_0$ is not well constrained by the current observational data. Hence, we introduced a Gaussian prior for $R_0$ as follows.
\begin{equation}
 Prior(R_0)= \frac{1}{\sqrt{2 \pi \sigma_{\rm R_0,prior}}}  \exp\left(-\frac{(R_0-R_{0,prior})^2}{2 \sigma_{R_0,prior}^2}\right),
\end{equation}
where we set $R_{0,prior}=8.3$ and $\sigma_{R_0,prior}=0.45$ from \citep{rdggb16}.

\section{Preliminary results}

Using the above method, we fit the Cepheids kinematics data. We used the data with the velocity errors less than 10 km~s$^{-1}$ and the vertical height, $|z-z_{\odot}|<0.2$ kpc. In addition, we select the data with the distance less than 4 kpc. Total number of stars used are 176.  The marginalised probability distribution is shown in Fig.~\ref{fig-mcmcall}. The results are summarised in Table~\ref{tab:MCMC-res}.

The results are similar to \citet{vvb17a} who used Cepheids cataglogue data with TGAS proper motion. 

Trial with $h_{\sigma}=4$ provides similar results. 

The differences in the results for the samples of $l<180$~deg and $l>180$~deg are consistent within their errors in the case if no $V_{\rm los,sys}$ fitting. 

The systematic line-of-sight motion is detected with $V_{\rm los,sys}=-2.9$~km~s$^{-1}$. However, if we use only $l>180$~deg data, it becomes a positive value. 

Young Cepheids with $\log P>0.8$ seems to show the smallest inward solar radial velocity, $V_{R,\odot}$, which is significantly smaller than the one from the older Cepheids with $\log<0.8$. This trend is also seen in \citet{vvb17a}.


\begin{table}
\centering
 \caption{Results of the MCMC fitting}
 \label{tab:MCMC-res}
 \begin{tabular}{lcccc}
  \hline
                                        & All                 &  no $V_{\rm los,sys}$ fit & $\log P>0.8$ & $\log P<0.8$ \\
 \hline
 $V_{\rm c}(R_0)$            & $245.8\pm  12.3$ & $249.9\pm  12.3$ & $244.5\pm  13.8$ & $247.9\pm  14.5$ \\
 
 $V_{\phi,\odot}$              & $256.9\pm  12.4$ & $260.8\pm  12.5$ & $257.0\pm  14.1$ & $256.9\pm  14.7$ \\

 $V_{R,\odot}$                 & $ -8.0\pm   1.0$    & $ -8.1\pm   1.0$    & $ -6.7\pm   1.5$ & $ -9.4\pm   1.4$ \\

 $\sigma_{\rm R}(R_0)$   & $ 13.2\pm   0.9$   & $ 13.2\pm   0.9$   & $ 13.4\pm   1.6$ & $ 12.7\pm   1.1$ \\

 $X^2$                             & $  0.9\pm   0.2$   & $  0.9\pm   0.2$    &   $  1.2\pm   0.5$ & $  0.9\pm   0.3$ \\

 $R_0$                             &  $  8.6\pm   0.4$    & $  8.7\pm   0.4$    & $  8.6\pm   0.4$ & $  8.5\pm   0.4$ \\
  
 $dV_{\rm c}(R_0)/dR$     & $ -2.9\pm   0.9$    & $ -3.3\pm   0.9$   & $ -3.0\pm   1.3$ & $ -4.1\pm   1.2$ \\

 $V_{\rm los,sys}$            & $ -2.8\pm   1.0$    & $-$                       &  $-$                       &  $-$                       \\ 

 N                                      & 176                       & 176                     &  87                         & 89 \\
 
\hline
\end{tabular}
\end{table}

% with Vlos,sys on
%\begin{table}
%\centering
% \caption{Results of the MCMC fitting}
% \label{tab:MCMC-res}
% \begin{tabular}{lcccccc}
%  \hline
%                                        & All                 &  no $V_{\rm los,sys}$ fit & $l<180$                &   $l>180$ &           
% $\log P>0.8$ & $\log P<0.8$ \\
% \hline
% $V_{\rm c}(R_0)$            & $245.8\pm  12.3$ & $249.9\pm  12.3$ &  $238.4\pm  19.1$ & $245.6\pm  18.3$ &
%  $242.4\pm  13.8$ & $245.2\pm  14.1$ \\
% $V_{\phi,\odot}$              & $256.9\pm  12.4$ & $260.8\pm  12.5$ &  $248.3\pm  19.4$ & $251.8\pm  18.6$  &
%  $254.9\pm  14.1$ & $254.9\pm  14.3$ \\
% $V_{R,\odot}$                 & $ -8.0\pm   1.0$    & $ -8.1\pm   1.0$    &  $ -7.5\pm   2.7$    & $ -7.6\pm   2.7$   &
%  $ -6.3\pm   1.5$    & $ -9.6\pm   1.3$ \\
% $\sigma_{\rm R}(R_0)$   & $ 13.2\pm   0.9$   & $ 13.2\pm   0.9$   &  $ 11.7\pm   1.6$    & $ 13.0\pm   1.1$    &
%  $ 13.5\pm   1.6$   & $ 12.6\pm   1.1$ \\
% $X^2$                             & $  0.9\pm   0.2$   & $  0.9\pm   0.2$    &   $  1.8\pm   0.8$     &  $  0.7\pm   0.2$  &
%  $  1.1\pm   0.5$    & $  0.9\pm   0.3$ \\
% $R_0$                             &  $  8.6\pm   0.4$    & $  8.7\pm   0.4$    &  $  8.5\pm   0.4$     & $  8.3\pm   0.4$   &
%   $  8.5\pm   0.4$    & $  8.5\pm   0.4$ \\
% $V_{\rm los,sys}$            & $ -2.8\pm   1.0$    & $-$                       & $ -5.2\pm   2.1$       & $  2.8\pm   2.1$   &
%   $ -2.7\pm   1.5$    & $ -2.7\pm   1.3$ \\
% $dV_{\rm c}(R_0)/dR$     & $ -2.9\pm   0.9$    & $ -3.3\pm   0.9$   & $ -2.9\pm   1.4$      &  $ -3.6\pm   1.1$ &
%   $ -2.8\pm   1.3$    & $ -3.5\pm   1.2$ \\
% N                                      & 176                       & 176                     & 83                           &   93 &
%  87                         & 89 \\
%\hline
%\end{tabular}
%\end{table}


\begin{figure}
\leavevmode
\includegraphics[width=\hsize]{MCMC-d4all}
\caption{\small
Marginalised probability distribution of the model parameters. All 176 data are used.
}
\label{fig-mcmcall}
\end{figure}

 



\footnotesize

\setlength{\baselineskip}{0pt}
%\setlength{\parskip}{-0.pt}
\setlength{\bibsep}{0pt}
\bibliography{/Users/dkawata/work/paper/dkref}

\end{document}
